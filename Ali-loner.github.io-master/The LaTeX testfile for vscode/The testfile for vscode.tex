\documentclass[a4paper]{article}
\usepackage[margin=1in]{geometry}%设置边距,符合Word设定
\usepackage{ctex}
\usepackage{setspace}
\usepackage{amsmath }
\usepackage{lipsum}
\usepackage{graphicx}%插入图片
\graphicspath{{Figures/}}%文章所用图片在当前目录下的 Figures目录
\setcounter{tocdepth}{5}
\setcounter{secnumdepth}{5}
\usepackage{hyperref} % 对目录生成链接,注:该宏包可能与其他宏包冲突,故放在所有引用的宏包之后
\hypersetup{colorlinks = true,  % 将链接文字带颜色
	bookmarksopen = true, % 展开书签
	bookmarksnumbered = true, % 书签带章节编号
	pdftitle = This is a testfile for vscode, % 标题
	pdfauthor =Ali-loner} % 作者
\bibliographystyle{plain}% 参考文献引用格式
\newcommand{\upcite}[1]{\textsuperscript{\cite{#1}}}

\renewcommand{\contentsname}{\centerline{Contents}} %经过设置word格式后,将目录标题居中


\title{\heiti\zihao{2} This is a testfile for vscode}
\author{\songti Ali-loner}
\date{2020.08.02}


\begin{document}
	\maketitle
	\thispagestyle{empty}

\begin{abstract}
	\lipsum[2]
\end{abstract}

\tableofcontents

\section{VSC plug in}
colonize

\section{This is a section}
Hello world! Hello Ali! As shown in figure \ref{1}
\begin{figure}[htbp]
	\centering
	\includegraphics[scale=0.2]{Ali.jpg}
	\caption{this is Sihan Cao}
	\label{1}
\end{figure}

\section{Molecular Dynamics}
Classical mechanics can not be the whole story.
Statistical Mechanics, (some system that not seem to go to the lowest energy)
\bigskip
Today I watch 46:43
\bigskip

\subsection{Probability Theory:}
\subsection{Probability Distribution}
All Probability Distribution must be normalized: sum over all possible outcomes must be "1"!
flip coin is a discrete variable (The outcome only has finite value), but in molecular dynamics we mainly think of continous variable.
\subsubsection{Normalization for continous variable}:
$\int_{-\infty}^{+\infty} p(x)dx = 1$
\subsubsection{Expected Value (aka Mean value, first moment of the distribution)}
\begin{equation}
	\langle X \rangle=\int_{-\infty}^{+\infty}xp(x)dx
\end{equation}
The $n_{th}$ order moment can be calculated through:
\begin{equation}
	\langle X^{n} \rangle = \int_{-\infty}^{+\infty}x^{n}p(x) dx
\end{equation}

\subsubsection{Statistical Property}
Variance: "cumulant"
\begin{equation}
	Var(X)=\langle X^{2} \rangle - {\langle X \rangle}^{2}
\end{equation}

Standard Deviation: Not cumulant, but has same unit as X
\begin{equation}
	std(x)=\sqrt{Var(X)}
\end{equation}

\subsection{Lattice Model}
Space is discretizied, and each discrete cell can hold 0 or 1 particle.
Microstate <---> Combinations

For the probablistic mechanics, the multi-cimponents system go to the (macro) state with the highest multiplicity (combos)
这句话是测试能否进行引用及支持中文\upcite{1}。

\section{Machine Learning and artifical intelligence for engineers}
\subsection{Lecture}

\subsubsection{Lecture 2}
\paragraph{Linear Regression}
To measure the error, just calculate the difference between the prediction result derived from the hypothesis, and calculate the difference between result and the ground truth. $d_{1}=\theta x^{(1)}-y^{(1)}, d_{2}=\theta x^{(2)}-y^{(2)}, \dots, d_{m}=\theta x^{(m)}-y^{(m)}$. We should sqaure it otherwise it will offset each other.
\newline
We calculate the distance, square it, and sum it to get our "OBJECTIVE FUNCTION", So the objective function (I think one type of objective function):
\begin{equation}
	\sum_{i=1}^{m}(y^{i}-h(x^{i}))^{2}
\end{equation}
The m is the number of training data point in our dataset.
\newline
所以,目标函数的意义就是,衡量一个拟合函数表现得好坏的工具。并且我们的参数是根据优化这个函数来计算得到的。So our goal is to parametrize it (参数化), 调整 $\theta$ 来优化这个目标函数。

\subsubsection{Lecture 3}
Gradient decsent 

All samples. 
\paragraph{Stichastic Gradient Descent} SGD (Stochastic Gradient Descent), don't sum all the samples, just do it one by one.
Stochastic (S) comes from 

\paragraph{Epoch} one epoch is go through all the data points from 1 to m.
When to stop training: the cost function.

\paragraph{Batch Gradient Descent}
\paragraph{mini Batch Gradient Descent}
One mini batch is one epoch. <<deeplearning.ai>> \url{deeplearning.ai} 

\paragraph{Cost function:}
the landsacpe is settle (The cost function is the same)
\paragraph{Evaluation matrices:}
SSE, sum square error (sum of square error for each sample)
MSE (Mean Square Error), devide SSE by m, which is the data points you have. 
\paragraph{Training and Test Set:}

\subsubsection{Lecture 4}
Callback: cost function, training set, test set.
\newline
today just go through probablity and Stichastic.
\paragraph{The quizz} interview question.
Extra point!!!
\paragraph{Joint Probability}
Multiple events occur at the same time.
\paragraph{conditional probablity}
\begin{equation}
	P(A|B)=\frac{P(A,B)}{P(B)}
\end{equation}
\paragraph{Product Rule}
$P(A_{1},A_{2}, \dots, A_{n})$ each event is like a data point in the machine learning problem.
\paragraph{conditional probability}

\paragraph{Bayes Rule}
One of the most important tool in Machine Learning. It talks about "OBSERVATION",
\begin{equation}
	P(reason|observation)=P(observation|reason)\cdot \frac{P(reason)}{P(observation)}
\end{equation}
我知道原因导致结果的概率,现在想计算观察到了结果,是由于这个原因导致可能性。
\newline
$P(data|observed)$n


\section{Probability and Estimation Method for Engineering System}
\subsection{Lecture}
\subsubsection{Lecture 3}
\paragraph{probability function}
\paragraph{Total Probability and Bayes' formula}
Conditional Distributions are distributions
conditional expectationa and variance
\subparagraph{Hw}
Independent event diagram

\section{Computer Version}
\subsection{Rotation Matrix}
$$
\begin{bmatrix}
	{\cos\theta} & {-\sin\theta} \\
	{\sin\theta} & {\cos\theta}
\end{bmatrix}	
$$
For the linear transformation we only need to care about 
$$
\begin{bmatrix}
	1 \\
	0
\end{bmatrix}
$$
which is the x-axis unit vector of the original coordinate and
$$ 
\begin{bmatrix}
	0 \\
	1
\end{bmatrix}
$$
which is the y-axis unit vector of the original coordinate.
Just draw a circle, and calculate the coordinate of the unit vector after rotation. The coordinate for the x-axis unit vector is the first column of the rotation matrix, and the y-axis is the second column.
\bigskip
To use the rotation Matrix it's just like:
$$
\begin{bmatrix}
	x^{1}\\
	y^{1}
\end{bmatrix}
=A
\begin{bmatrix}
	x\\
	y
\end{bmatrix}
$$
\subsection{Lecture}
\subsubsection{Lecture1}
Think image as a function, a color image is just like:
$$
f(x, y)=
\begin{bmatrix}
	r(x,y)\\
	g(x,y)\\
	b(x,y)
\end{bmatrix}
$$
\bigskip
For the image Processing, there are point operation and neighborhood operation.


\section{Molecular Dynamics}
\subsection{Lecture}
\subsubsection{Lecture 3}

\bigskip
\section{MD Homework Part}
Stirling's approximation 
\begin{equation}
	N! = (\frac{N}{e})^{N}
	\ln N! \approx N\ln N- N
\end{equation}

\subsection{Problem 1}
use molecular dynamics method to investigate the fracture toughness of calcium-silicate-hydrate (C-S-H), which is the binding phase of concrete and responsible for mechanical property of concrete.

did a research using molecular dynamics method to investigate the fracture toughness of calcium-silicate-hydrate (C-S-H), which is the binding phase of concrete and responsible for its mechanical property. The property of C-S-H plays a pivotal role on the mechanical property of concrete, and MD could help us take out experiment on in the atomic scale. 
The result shows that in the atomic scale, the C-S-H shows a ductile behavior. So it is not proper to implement method which is built based on linear elastic fracture mechanics. Another cool thing is that the atomic level investigation about C-S-H fracture performance could be used in the upscaling approcah to help engineers in much larger scale like cement paste design.

The first thing that is interesting to me is that molecular dynamics could help us get many useful parameters like fracture toughness just rely computer without doing any experiment. I think this make MD a very cool tool for engineers and scientists. Even we may not have real material or enough funding to take out complex experiment, we can still do many things just rely on a good computer.

It is also interesting to see the role that statistics plays in the molecular simulation. In this paper, authors use pair distribution function (PDF) to validate the realistic model of C-S-H used in the simulation via comparing the pdf with result from experiment.

This paper also mentioned that the results derived from MD could be used in a upscale way to mesoscale macroscale, which is also called bottom-up approach. The detail plan is not included in this paper, but I am very interested in how could we use the MD result to help people in larger scale, which in my mind has a more direct relationship with our real life.

\subsection{Problem 5}
Lennard Jones law: $4 \epsilon [(\frac{\sigma}{r})^{12}-(\frac{\sigma}{r})^{6}]$
For a FCC material $U(d)=\frac{1}{2} 4 \epsilon $
\newline
The $r^{12}$ term is the short term repulsive term (describe Pauli Repulsion), and the $r^{6}$ is the long term attractive term (describe van der Waals force or dispersion force).

\subsubsection{(a)}
The sum of the energy of the material could be written as: 
\begin{equation}
	U=4\epsilon [(\frac{\sigma}{r})^{12}-(\frac{\sigma}{r})^{6}]=\frac{1}{2}N 4\epsilon (\frac{(\sigma)^{12}}{(d)^{12}}A-\frac{(\sigma)^{6}}{(d)^{6}}B)
\end{equation}

To calculate A and B:
\begin{equation}
	A=2*(\frac{\sigma^{12}}{d^{12}}) + 2*(\frac{\sigma^{12}}{(2d)^{12}}) + 2*(\frac{\sigma^{12}}{(3d)^{12}})+...=2.0005 (\frac{\sigma^{12}}{d^{12}})
\end{equation}
\begin{equation}
	B=2*(\frac{\sigma^{6}}{d^{6}}) + 2*(\frac{\sigma^{6}}{(2d)^{6}}) + 2*(\frac{\sigma^{6}}{(3d)^{6}})+2*(\frac{\sigma^{6}}{(4d)^{6}})+... = 1.0173 (\frac{\sigma^{6}}{d^{6}})
\end{equation}

Then,
\begin{equation}
	U=\frac{1}{2}N 4 \epsilon [2.0005 \frac{\sigma^12}{d^12}-1.0173 \frac{\sigma^6}{d^6}]
\end{equation}
To calculate the equilibirium space d, 
\begin{equation}
	\frac{\mathrm{d} U}{\mathrm{d}r}=0
\end{equation}

So the $d$ is $\sqrt[6]{\frac{2 \times 2.0005}{1.0173}} \sigma=1.2564 \sigma$

The d just depend on length scale $\sigma$, but has no relationship with energy scale $\epsilon$

\subsubsection{b}
If there is no energy dissipation, the total system will vibrate like a wave. 

\subsubsection{c}
The total energy of the system is $U=2N \epsilon [A\frac{\sigma^{12}}{r^{12}}-B\frac{\sigma^{6}}{r^{6}}]$, so $\frac{\mathrm{d^{2}}u}{\mathrm{d}r^{2}}=2N \epsilon [A \sigma^{12}(12 \times 13 r^{-14})-B \sigma^{6}(42 r^{-8})]=2N\epsilon [2.0005\sigma^{12} \times 156 \times(1.2564 \sigma)^{-14}-1.0173\sigma^{6}\times 42 \times (1.2564 \sigma)^{-8}]$
So the effective spring constant equals:$11.7938N\epsilon \sigma^{-2}$

\subsection{Problem 6}
The weight of one mole water is 18 g/mol, and the density of water is 1g/mL.
\subsubsection{diameter of 1 nm:} The volume of the water drop is $\frac{4}{3}\pi r^{3}$, so the volume of the drop is $\frac{4}{3}\pi \cdot (1 nm)^{3}$. The density is 1g/mL, which could be written as $1g/cm^{3}=1g/(10^{6}nm)^{3}$. So the number of molecules can be calculated through the following formulation:
\begin{equation}
	Mole=\frac{4}{3}\pi \cdot (1 nm)^{3} \cdot 1g/(10^{6}nm)^{3} \div 18 g/mol=2.327 \times 10^{-19} mole 
\end{equation}
To get the number of molecules just multiply the mole with Avogrado's number:
\begin{equation}
	Number =2.327 \times 10^{-19} mole \times 6.022 \times 10^{23}=140132 \approx 1.4 \times 10^{5}
\end{equation}

\subsubsection{diameter of 1$\mu$ m}
Same method just changed the volume of the drop:
\begin{equation}
	\frac{4}{3}\pi \cdot (10^{3} nm)^{3} \cdot 1g/(10^{6}nm)^{3} \div 18 g/mol \times 6.022 \times 10^{23} \approx 1.4 \times 10^{14}
\end{equation}

\subsubsection{diameter of 1 mm}
\begin{equation}
	\frac{4}{3}\pi \cdot (10^{6} nm)^{3} \cdot 1g/(10^{6}nm)^{3} \div 18 g/mol \times 6.022 \times 10^{23} \approx 1.4 \times 10^{23}
\end{equation}

\subsection{b}
Note that there are "A" water molecules. There are 3A atoms in the system. For the first water molecule, its atoms could have (3A-3) interactions. For the second water molecule, its atoms have (3A-6). Based on this pattern, the total pairs is:
\begin{equation}
	3(A-1) \cdot 3(A-2) \cdot \dots \cdot 3(2) \cdot 3(1)=3^{A-1} \times (A-1)!
\end{equation}
So, for the waterdrop with 1 nm, the number of pairs is; For the mm pair, the number of pair is $3^{1.4 \times 10^{23}} \times (1.4 \times 10^{23}-1)!$

\subsection{c}
When I try to print the number, there is a overflow error in python.

\subsection{Problem 7}
\subsubsection{(a)}
There are total 100 votes remained. If A wants to win the electration, A has to get 40 votes at least. So there are following possible situation that A will win: $A_{i}:\text{A gets i votes and win,} i={40,50,60, \dots, 100}$.
$$
\begin{cases}
	A_{40}=C_{4}^{2}+C_{4}^{1} \cdot C_{2}^{2}=10 \\
	A_{50}=C_{2}^{1} \cdot C_{4}^{2}=12 \\
	A_{60}=C_{2}^{0} \cdot C_{4}^{3} + C_{2}^{2} \cdot C_{4}^{2}=10 \\
	A_{70}=C_{2}^{1} \cdot C_{4}^{3}=8 \\
	A_{80}=C_{2}^{0} \cdot C_{4}^{4} + C_{2}^{2} \cdot C_{4}^{3}=5 \\
	A_{90}=C_{2}^{1} \cdot C_{4}^{4}=2 \\
	A_{100}=C_{2}^{2} \cdot C_{4}^{4}=1
\end{cases}
$$
So, $W_{A}=\sum_{i=40}^{100} A_{i}=48$

\subsubsection{(b)}
Same idea like (a), B has to get at least 70 votes to win the electration, so there are 4 situations that B can win: $B_{i}:\text{B gets i votes and win,} i={70, 80 \dots, 100}$
$$
\begin{cases}
	B_{70}=C_{2}^{1} \cdot C_{4}^{3}=8 \\
	B_{80}=C_{2}^{0} \cdot C_{4}^{4} + C_{2}^{2} \cdot C_{4}^{3}=5 \\
	B_{90}=C_{2}^{1} \cdot C_{4}^{4}=2 \\
	B_{100}=C_{2}^{2} \cdot C_{4}^{4}=1
\end{cases}
$$
So, $W_{B}=\sum_{i=70}^{100} B_{i}=16$

\subsubsection{(c)}
Note A beats B as event C, then $P(C)=\frac{W_{A}}{W_{A}+W_{B}}=\frac{3}{4}=75 \%$

\subsection{Problem 8}
\subsubsection{a}
For one lattice, the combination is $C_{N}^{n}$. Because the system has two lattices, so the total number of combinations is $C_{N}^{n} \cdot C_{N}^{n} = (C_{N}^{n})^2$
\subsubsection{b}
The total number of combinations is $C_{2N}^{2n}$
The $W_{A}$ could be writeen as $\frac{N!}{n!(N-n)!}\cdot\frac{N!}{n!(N-n)!}$, and the $W_{B}$ could be written as $\frac{(2N)!}{(2n)!(2N-2n)!}$ Based on the form of sterling approximation, when N is really large is $N! = N^{N}e^{-N}$, So the ratio between $W_{A}$ and $W_{B}$ could be calculated through:
\begin{equation}
	\frac{W_{B}}{W_{A}}=\frac{(2N)!}{(2n)!(2N-2n)!}\cdot \frac{n! n!}{(2n)!}\cdot \frac{(N-n)!(N-n)!}{(2N-2n)!}
\end{equation}
We could use $\frac{N!N!}{(2N)!}$ as an example:
\begin{equation}
	\frac{(\frac{2N}{e})^{2N}}{(\frac{N}{e})^{2N}}=2^{2N}
\end{equation}
So the second term is just using $n$ to replace the $N$ in the previous equation, so it is $\frac{1}{2^{2n}}$ and the third term is $\frac{1}{2^{2N-2n}}$. 
\newline
So the result is the total Product of the three term:
\begin{equation}
	2^{2N} \cdot \frac{1}{2^{2n}} \cdot \frac{1}{2^{2N-2n}}=1
\end{equation}
\subsubsection{c}
Based on the previous question, the entropy difference could be calculated as following:
\begin{equation}
	\Delta S=k \ln (\frac{W_{B}}{W_{A}})=k \ln(1) =0
\end{equation}
\newline
\subsection{Problem 9}
Proving this equation is larger than 0 equals to prove:
\begin{equation}
	\int _{-\infty}^{+\infty}\frac{\partial}{\partial t}[\rho(x,t)\ln \rho(x,t)]dx \leq 0
\end{equation}
The equation can be transformed to (ignored D):
\begin{equation}
	\int_{-\infty}^{+\infty}(\ln \rho +1)\frac{\partial^{2}\rho}{\partial x^{2}}dx =
	\int_{-\infty}^{+\infty}(\ln \rho +1) d(\frac{\partial \rho}{\partial x})
\end{equation}
Based on the intergation by parts, it could be written as:
\begin{equation}
	(\ln \rho +1)\cdot \frac{\partial \rho}{\partial x}|_{-\infty}^{+\infty}-\int_{-\infty}^{+\infty}\frac{\partial \rho}{\partial x}d(\ln \rho +1)=\\
	\int_{-\infty}^{+\infty}\frac{\partial \rho}{\partial x}d(\ln \rho +1)=\\
	-\int_{-\infty}^{+\infty}(\frac{\partial \rho}{\partial x})^{2}\frac{1}{\rho}dx
\end{equation}
which is obvious less than 0, because all things in intergation is positive but there is a minus outside. So this also obey second law.

\section{概率论作业}

\subsection{Problem 3}
\subsubsection{(a)}
To prove $A\bot B$, we just need to prove the following equation:
\begin{equation}
	P(AB)=P(A)\cdot P(B)
\end{equation}
The event A could has two corresponding outcomes: (1) One day is rainy, and other two days are sunny; (2) One day is sunny, and other two days are rainy. So $P(A)$ could be calculated through:
\begin{equation}
	P(A)=C_{3}^{1} \times \frac{1}{2} \times \frac{1}{2}^{2} \times 2=\frac{3}{4}
\end{equation}
To calculate $P(B)$, event B could also be devided into two outcomes: (1)one day is rainy; (2) all days are sunny. So $P(B)$ can be derived from:
\begin{equation}
	P(B)=(\frac{1}{2})^{3}+C_{3}^{1}\times \frac{1}{2} \frac{1}{2}^{2}=\frac{1}{2}
\end{equation}

$A \cap B$ means exactly one day is rainy, and two other days are sunny, so:
\begin{equation}
	P(A,B)=C_{3}^{1} \times \frac{1}{2} \times \frac{1}{2}^{3}=\frac{3}{8}
\end{equation}
So, we have:
\begin{equation}
	P(A,B)=P(A)\cdot P(B)
\end{equation}
So, we can say that $A \bot B$.

we could prove that $P(B,C)=P(B)\cdot P(C)$ to prove that $B \bot C$.
To calculate $P(C)$, we could devide event C into two outcomes: (1) All sunny days; (2) All rainy days.
\begin{equation}
	P(C)=\frac{1}{2}^{3}+\frac{1}{2}^{3}=\frac{1}{4}
\end{equation}

The event $B \cap C$ means all days are sunny days, so:
\begin{equation}
	P(B,C)=\frac{1}{2}^{3}=\frac{1}{8}
\end{equation}

So, 
\begin{equation}
	P(B)\cdot P(C)=\frac{1}{4} \times \frac{1}{2}=\frac{1}{8}=P(B,C)
\end{equation}
Then we can say that $B \bot C$.
\subsubsection{(b)}
Event $A \bot C$ want all three days have the same weather but at the same time include at least one rainy day and sunny day, which is impossible. So:
\begin{equation}
	P(A,C)=0 \neq P(A)\cdot P(C)=\frac{3}{16}
\end{equation}
So A and C are not independent event. It is reasonable, because A and C could not happen at the same time, so we definetly can learn some thing from one event about the other event.

\subsection{Problem 4}
\subsubsection{(a)}
According to the condition given in the problem statement, we have: $P(S)=0.2 \%, P(O|S)=25 \%, P(S^{c}=1-P(S)-98 \%, P(O|S^{c}=98 \%))$. Based on those, we could derive the joint probability:
\begin{equation}
	P(O,S)=P(S)\cdot P(O|S)=2 \% \times 25 \% =0.5 \%\\
	P(O,S^{c})=P(S^{c}) \cdot p(O|S^{c})=98 \times 0.3 \%=0.294\%
\end{equation}
According to the total probablity, the marginal probability of outage is:
\begin{equation}
	P(O)=P(O,S)+P(O,S^{c})=0.794\%
\end{equation}

\subsubsection{(b)}
The conditional probability of strom occured when observed outage is:
\begin{equation}
	P(S|O)=\frac{P(S,O)}{P(O)}=\frac{0.5 \%}{0.794 \%}=62.97 \%
\end{equation}
\subsubsection{(c)}
\subsubsection{(d)}


\section{Latex usage}
The citation part of using latex: \url{https://zhuanlan.zhihu.com/p/25013341}.
在写文档之前的引用库部分,使用“usepackage{apacite},bibliographystyle{apacite}”。行文中要引用就使用citeA{konishi:1999ab}。在结束问当前添加bibliography{example},example是文献数据库的名字 (example.bib)


\bibliography{books}
\end{document}