\documentclass[a4paper]{article}
\usepackage[margin=1in]{geometry}%设置边距,符合Word设定
\usepackage{ctex}
\usepackage{setspace}
\usepackage{amsmath }
\usepackage{lipsum}
\usepackage{graphicx}%插入图片
\graphicspath{{Figures/}}%文章所用图片在当前目录下的 Figures目录
\setcounter{tocdepth}{5}
\setcounter{secnumdepth}{5}
\usepackage{hyperref} % 对目录生成链接,注:该宏包可能与其他宏包冲突,故放在所有引用的宏包之后
\hypersetup{colorlinks = true,  % 将链接文字带颜色
	bookmarksopen = true, % 展开书签
	bookmarksnumbered = true, % 书签带章节编号
	pdftitle = This is a testfile for vscode, % 标题
	pdfauthor =Ali-loner} % 作者
\bibliographystyle{plain}% 参考文献引用格式
\newcommand{\upcite}[1]{\textsuperscript{\cite{#1}}}

\renewcommand{\contentsname}{\centerline{Contents}} %经过设置word格式后,将目录标题居中


\title{\heiti\zihao{2} This is a testfile for vscode}
\author{\songti Ali-loner}
\date{2020.08.02}


\begin{document}
	\maketitle
	\thispagestyle{empty}

\begin{abstract}
	\lipsum[2]
\end{abstract}

\tableofcontents

\section{This is a section}
Hello world! Hello Ali! As shown in figure \ref{1}
\begin{figure}[htbp]
	\centering
	\includegraphics[scale=0.2]{Ali.jpg}
	\caption{this is Sihan Cao}
	\label{1}
\end{figure}

\section{Molecular Dynamics}
Classical mechanics can not be the whole story.
Statistical Mechanics, (some system that not seem to go to the lowest energy)
\bigskip
Today I watch 46:43
\bigskip

\section{Probability Theory:}
\subsection{Probability Distribution}
All Probability Distribution must be normalized: sum over all possible outcomes must be "1"!
flip coin is a discrete variable (The outcome only has finite value), but in molecular dynamics we mainly think of continous variable.
\subsubsection{Normalization for continous variable}:
$\int_{-\infty}^{+\infty} p(x)dx = 1$
\subsubsection{Expected Value (aka Mean value, first moment of the distribution)}
\begin{equation}
	\langle X \rangle=\int_{-\infty}^{+\infty}xp(x)dx
\end{equation}
The $n_{th}$ order moment can be calculated through:
\begin{equation}
	\langle X^{n} \rangle = \int_{-\infty}^{+\infty}x^{n}p(x) dx
\end{equation}

\subsubsection{Statistical Property}
Variance: "cumulant"
\begin{equation}
	Var(X)=\langle X^{2} \rangle - {\langle X \rangle}^{2}
\end{equation}

Standard Deviation: Not cumulant, but has same unit as X
\begin{equation}
	std(x)=\sqrt{Var(X)}
\end{equation}

\subsection{Lattice Model}
Space is discretizied, and each discrete cell can hold 0 or 1 particle.
Microstate <---> Combinations

For the probablistic mechanics, the multi-cimponents system go to the (macro) state with the highest multiplicity (combos)
这句话是测试能否进行引用及支持中文\upcite{1}。

\section{Machine Learning and artifical intelligence for engineers}
\subsection{Lecture}
\subsubsection{Lecture 3}
Gradient decsent 

All samples. 
\paragraph{Stichastic Gradient Descent} SGD (Stochastic Gradient Descent), don't sum all the samples, just do it one by one.
Stochastic (S) comes from 

\paragraph{Epoch} one epoch is go through all the data points from 1 to m.
When to stop training: the cost function.

\paragraph{Batch Gradient Descent}
\paragraph{mini Batch Gradient Descent}
One mini batch is one epoch. <<deeplearning.ai>> \url{deeplearning.ai} 

\paragraph{Cost function:}
the landsacpe is settle (The cost function is the same)
\paragraph{Evaluation matrices:}
SSE, sum square error (sum of square error for each sample)
MSE (Mean Square Error), devide SSE by m, which is the data points you have. 
\paragraph{Training and Test Set:}


\section{Computer Version}
\subsection{Rotation Matrix}
$$
\begin{bmatrix}
	{\cos\theta} & {-\sin\theta} \\
	{\sin\theta} & {\cos\theta}
\end{bmatrix}	
$$
For the linear transformation we only need to care about 
$$
\begin{bmatrix}
	1 \\
	0
\end{bmatrix}
$$
which is the x-axis unit vector of the original coordinate and
$$ 
\begin{bmatrix}
	0 \\
	1
\end{bmatrix}
$$
which is the y-axis unit vector of the original coordinate.
Just draw a circle, and calculate the coordinate of the unit vector after rotation. The coordinate for the x-axis unit vector is the first column of the rotation matrix, and the y-axis is the second column.
\bigskip
To use the rotation Matrix it's just like:
$$
\begin{bmatrix}
	x^{1}\\
	y^{1}
\end{bmatrix}
=A
\begin{bmatrix}
	x\\
	y
\end{bmatrix}
$$
\subsection{Lecture}
\subsubsection{Lecture1}
Think image as a function, a color image is just like:
$$
f(x, y)=
\begin{bmatrix}
	r(x,y)\\
	g(x,y)\\
	b(x,y)
\end{bmatrix}
$$
\bigskip
For the image Processing, there are point operation and neighborhood operation.


\bigskip
\section{Homework Part}
Stirling's approximation 
\begin{equation}
	N! = (\frac{N}{e})^{N}
	\ln N! \approx N\ln N- N
\end{equation}
\subsection{Problem 5}
Lennard Jones law: $4 \epsilon [(\frac{\sigma}{r})^{12}-(\frac{\sigma}{r})^{6}]$
For a FCC material $U(d)=\frac{1}{2} 4 \epsilon $
\newline
The $r^{12}$ term is the short term repulsive term (describe Pauli Repulsion), and the $r^{6}$ is the long term attractive term (describe van der Waals force or dispersion force).

\subsubsection{(a)}
The sum of the energy of the material could be written as: 
\begin{equation}
	U=4\epsilon [(\frac{\sigma}{r})^{12}-(\frac{\sigma}{r})^{6}]=\frac{1}{2}N 4\epsilon (\frac{(\sigma)^{12}}{(d)^{12}}A-\frac{(\sigma)^{6}}{(d)^{6}}B)
\end{equation}

To calculate A and B:
\begin{equation}
	A=2*(\frac{\sigma^{12}}{d^{12}}) + 2*(\frac{\sigma^{12}}{(2d)^{12}}) + 2*(\frac{\sigma^{12}}{(3d)^{12}})+...=2.0005 (\frac{\sigma^{12}}{d^{12}})
\end{equation}
\begin{equation}
	B=2*(\frac{\sigma^{6}}{d^{6}}) + 2*(\frac{\sigma^{6}}{(2d)^{6}}) + 2*(\frac{\sigma^{6}}{(3d)^{6}})+2*(\frac{\sigma^{6}}{(4d)^{6}})+... = 1.0173 (\frac{\sigma^{6}}{d^{6}})
\end{equation}

Then,
\begin{equation}
	U=\frac{1}{2}N 4 \epsilon [2.0005 \frac{\sigma^12}{d^12}-1.0173 \frac{\sigma^6}{d^6}]
\end{equation}
To calculate the equilibirium space d, 
\begin{equation}
	\frac{\mathrm{d} U}{\mathrm{d}r}=0
\end{equation}

So the $d$ is $\sqrt[6]{\frac{2 \times 2.0005}{1.0173}} \sigma=1.2564 \sigma$

The d just depend on length scale $\sigma$, but has no relationship with energy scale $\epsilon$

\subsubsection{b}
If there is no energy dissipation, the total system will vibrate like a wave. 

\subsubsection{c}
The total energy of the system is $U=2N \epsilon [A\frac{\sigma^{12}}{r^{12}}-B\frac{\sigma^{6}}{r^{6}}]$, so $\frac{\mathrm{d^{2}}u}{\mathrm{d}r^{2}}=2N \epsilon [A \sigma^{12}(12 \times 13 r^{-14})-B \sigma^{6}(42 r^{-8})]=2N\epsilon [2.0005\sigma^{12} \times 156 \times(1.2564 \sigma)^{-14}-1.0173\sigma^{6}\times 42 \times (1.2564 \sigma)^{-8}]$
So the effective spring constant equals:$11.7938N\epsilon \sigma^{-2}$

\subsection{Problem 8}
\subsubsection{a}
For one lattice, the combination is $C_{N}^{n}$. Because the system has two lattices, so the total number of combinations is $C_{N}^{n} \cdot C_{N}^{n} = (C_{N}^{n})^2$
\subsubsection{b}
The total number of combinations is $C_{2N}^{2n}$

\bibliography{books}
\end{document}